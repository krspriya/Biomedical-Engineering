\documentclass[12pt]{article}


\usepackage{geometry}
\geometry{
 a4paper,
 total={180mm,260mm},
 left=16mm,
 right=16mm,
 top=16mm,
 bottom=16mm,
 }

 
\usepackage{graphicx}
\graphicspath{{images/}}





\begin{document}




















\newpage
\centering
\section*{\Huge Acknowledgement}
\vspace{3mm}
\raggedright
\LARGE I am grateful to Department of Bio Medical Engineering for their proficient supervision of the term project on "Regenerative Medicine ”. I am very thankful to department for their guidance and support.\\
	
\raggedleft

\vspace{45mm}

Smriti Priya,
\\21111060,


1st semester,
\\ Biomedical Engineering
\\National Institute of Technology,Raipur

\raggedright 
\vspace{110mm}
Date of submission: 4th April , 2022
\newpage
\centering 
\section*{ \Huge Abstract }
\vspace{3mm}
\Large
\raggedright
Regenerative medicine has gotten a lot of attention in recent decades because it has the potential to overcome the constraints of direct transplantation, such as donor shortages and immunological problems. The rapid advancement in this field necessitates the rapid development of living materials, which are made up of live biological agents and can be combined with synthetic materials to satisfy the needs of regenerative medicine. 



\centering \section*{\Huge Introduction }
\Large

\raggedright
Regenerative medicine isn't simply a pipe dream; it's a reality right now. Many cell-based medicines and products are already on the market, and many more are being studied in patients. These items give us a glimpse into what the future holds for patient health and the economy. Regenerative medicine is a multidisciplinary field, and individuals involved must draw on the experience of a wide range of fields and stakeholders to maximise their chances of success. Collaboration between academia, industry, and physicians is critical for the regenerative medicine field's future development. This rapidly developing field of medicine, particularly cell-based therapeutics, has the potential to provide significant clinical advantages while also addressing critical unmet medical needs.

\begin{figure}[h]
\centering
\includegraphics[scale=0.6]{re.jpg}
\end{figure}
\newpage
\raggedright
\section*{\LARGE  What is tissue engineering? }
\Large
\raggedright
“Tissue engineering is an interdisciplinary field that applies the principles of engineering and the life sciences toward the development of biological substitutes that restore, maintain, or improve tissue function”
\vspace{3mm}
Tissue engineering combines cells, scaffolds, and growth factors to regenerate or replace damaged or diseased tissues, whereas regenerative medicine combines tissue engineering with other strategies to induce in vivo tissue/organ regeneration, such as cell-based therapy, gene therapy, and immunomodulation.

\section*{\LARGE  What is Biomaterials? }
\Large
\raggedright
A biological or synthetic substance that can be transplanted into body tissue as part of a medical device or utilised to replace an organ, bodily function, or other bodily function.\\
\vspace{3mm}
Engineered biomaterials are used in regenerative medicine to harness the human body's natural ability to repair and regenerate damaged tissues. Cellular response can be regulated to influence tissue repair by creating responsive biomaterials with the necessary biophysical and biochemical features.

\section*{\LARGE  What is  stem cells? }
\Large
\raggedright
Stem cells are the body's raw materials — cells from which all other cells with specialized functions are generated. Under the right conditions in the body or a laboratory, stem cells divide to form more cells called daughter cells.\\
\vspace{3mm}
Stem cell therapy, also known as regenerative medicine, promotes the repair response of diseased, dysfunctional or injured tissue using stem cells or their derivatives. It is the next chapter in organ transplantation and uses cells instead of donor organs, which are limited in supply.

\section*{\LARGE  What types of treatments are being developed? }
\Large
\raggedright
Researchers are working on treatments that stimulate previously irreparable tissues to heal themselves in order to regenerate damaged tissues in the body.\\
\vspace{3mm}
For example :
\begin{itemize}
\item Tissue engineering has been used to create skin and bladder cells that, once transplanted, stimulate the 
growth of bone, connective tissues, and knee cartilage.
\item Precursor cells, such as stem cells derived from adult tissues, are being investigated for their ability to generate engineered tissue.
\item Researchers have developed a method for maturing stem cells into mature bone cells that could be transplanted into a patient.
\end{itemize}

\section*{\LARGE How is regenerative medicine regulated in India? }
The Indian government has heavily promoted regenerative medicine research and development, as well as domestic innovation and business development initiatives. Together, these initiatives promise to usher in a new era of healthcare and public empowerment in India. \\
\vspace{3mm}
Several national and transnational collaborations have emerged to develop innovative capacity, most notably in stem cell and cord blood banking, gene therapy, tissue engineering, biomaterials, and 3D printing.\\
\vspace{3mm}
 However, there are still obstacles to achieving regulatory oversight, viable outputs, and equitable outcomes. More attention is needed for the governance of private cord blood banking, nanomaterials, and 3D bioprinting. A strong social contract is also required in healthcare in general, so that participation in regenerative medicine research and innovation is backed up by treatments that are widely available to all.

\section*{\LARGE Implications of Regenerative Medicine  }
\Large
\raggedright
Although regenerative medicine holds promise, safety and effectiveness are likely to vary depending on the therapy or product.Because the safety and efficacy of treatments derived from blood or bone marrow are well established, the FDA (Food and Drug Administration) has approved them.Patients and health care providers should be able to understand the risks and benefits of regenerative medicine therapies if they are evaluated using adequate evidence.\\
\vspace{3mm}
Patients may be unaware of which types of regenerative products and procedures are regulated by the FDA, and the agency does not maintain a comprehensive list of clinics or manufacturers of these products or treatments.
\vspace{3mm}
To protect the public, reliable scientific evidence of their safety and effectiveness will be required. Otherwise, patients would be exposed to unknown hazards, undermining consumer confidence in this burgeoning profession. The development of this evidence will be critical to realise regenerative medicine's promise.

\newpage



\centering \section*{\Huge Conclusion }
\Large
\raggedright
Patients with injuries, end-stage organ failure, or other clinical conditions may benefit from the use of regenerative medicine technologies. Patients with damaged or wounded organs can currently be treated with transplanted organs. However, there is a significant scarcity of donor organs, which is getting worse every year as the population ages and more people develop organ failure. Regenerative medicine and tissue engineering researchers are currently using cell transplantation, material science, and bioengineering principles to create biological alternatives that can restore and sustain normal function in diseased and wounded tissues.

The field of stem cells is also fast developing, bringing up new paths for this form of therapy. Therapeutic cloning and cellular reprogramming, for example, could one day provide an almost endless supply of cells for tissue engineering applications. While stem cells are still in the research stage, some tissue engineering-based medicines have already been approved for use in the clinic, demonstrating the future promise of regenerative medicine.




\centering \section*{\Huge References }
\Large
\raggedright  
1) $ https://www.ncbi.nlm.nih.gov/pmc/articles/PMC4664309/ $

\vspace{5mm}

2) $ https://en.wikipedia.org/wiki/Regenerative_medicine/  $ 

\vspace{5mm}

3) $ https://www.sciencedirect.com/topics/medicine-and-dentistry/  
\vspace{3mm} regenerative-medicine $




\end{document}